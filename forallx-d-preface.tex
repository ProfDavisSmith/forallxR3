\chapter{Preface}

This textbook concerns formal logic. Formal logic, as a field, is a subdiscipline within Philosophy which studies the relationship between assumptions (premises) and conclusions. It is `formal' in the sense that it concerns the \emph{form} or \emph{shape} of the assumptions/conclusions on an abstract level so the same logical structure can be used in many different unrelated contexts with the same level of certainty. The forms of these arguments are expressed through the use of \emph{formal languages}, namely Propositional Logic, Quantified Logic, and Modal Logics. Being formal, these languages are very useful in that it is impossible to be ambiguous. The study of such formal languages centers on entailment, that is, what follows from what; by better understanding entailment, we can better tell what will be a necessary consequence of a state of affairs or situation. We can, in a sense, predict the future. Studying these formal languages also gives us a better understanding of notions like possibility, necessity, validity, and soundness. Formal languages naturally have the ability to precisely define these notions using either the basic aspects of the language or the systems of deduction which the language naturally provides.

The importance of formal logic and these formal languages extends beyond Philosophy and regular reasoning/argumentation into other fields such as Mathematics and Computer Science. In Mathematics, formal languages are used to describe mathematical situations rather than normal ``everyday'' ones. For their proofs and discoveries, mathematicians are fundamentally interested in the consequences of definitions and assumptions and it is very important for them to establish these consequences (which they call ``theorems'') using extremely precise and rigorous methods so that they will hold regardless of the circumstances. Formal logic provides those methods. 

In computer science, formal logic is applied to describe the state and behaviours of computational systems, e.g., circuits, programs, databases, and the like. The methods in formal logic are used to establish whether a circuit is error-free, such as whether it returns the correct output given some set of inputs; whether a program does what it's intended to do; whether the structure of a database is consistent, whether the storage of the data is optimally compact, or if something is true of the data in it; and other areas of concern. In fact, the underlying frameworks for computer programming languages are based on the formal languages in formal logic. 

The book is divided into ten (10) modules. Module~\ref{ch.intro} introduces the topic and notions of logic in an informal way, without introducing a formal language yet.  Modules \ref{ch.symbolizing}--\ref{ch.plnd2} concern truth-functional languages. In it, sentences are formed from basic sentences using a number of connectives (`or', `and', `not', `if \dots then') which combine sentences into more complicated ones.  We discuss logical notions such as entailment in two ways: semantically, using the method of truth tables (in Module~\ref{ch.TruthTables}) and proof-theoretically, using a system of formal derivations (in Modules~\ref{ch.plnd1}--\ref{ch.plnd2}).  Modules \ref{ch.qlsymbolizing}--\ref{ch.qlnd2} deal with a more complicated language, that of first-order logic, called Quantified Logic. It includes, in addition to the connectives of truth-functional logic (called Propositional Logic), also names, predicates, identity, and quantifiers. These additional elements of the language make it much more expressive than Propositional Logic and able to handle far more complicated situations. We'll spend a fair amount of time investigating just how much one can express in it.  Again, logical notions for Quantified Logic are defined semantically, using interpretations, and proof-theoretically, using a more complex formal derivation system introduced in Modules \ref{ch.qlnd1} and \ref{ch.qlnd2}.  Module~\ref{ch.ml1} discusses the extension of PL by non-truth-functional operators for possibility and necessity: modal logic. In the appendices you'll find proofs concerning metalogical concepts such as \emph{soundness} and \emph{completeness}, a chapter outlining the link between computer circuits and Propositional Logic, a quick reference listing the rules and restrictions (if necessary) for the truth conditions and inference rules, and a glossary listing central terms used thoughout the book. Future editions of this book will include content and proof-systems for deviant logics, such as paraconsistent logic and other many-valued logics, content and proof-systems for alternative applications of Modal Logic (such as Temporal, Deontic, and Epistemic Logics), and a more in depth exploration of inductive reasoning and informal fallacies. 

\paragraph{Credits} This book is based on \forallx: \emph{Calgary}, by Aaron Thomas-Bolduc and Richard Zach which was based on a text originally written by P.~D. Magnus in the version revised and expanded by Tim Button. Through \forallx: \emph{Calgary}, it also includes some material by J.~Robert Loftis and some material by Robert Trueman (mostly content in Module~\ref{ch.ml1}). Davis A. Smith `remixed' and revised the material provided by Aaron Thomas-Bolduc and Richard Zach and then expanded upon it with his own original content, including but not limited to: Redefining an inference rule, adding a new equivalency rule, including more examples, reimagining the proof system for the Modal Logics (adding new inference rules and redesigning the `strict subproof' system for better implementation into Carnap.io), adding biographies for various prominate logicians, metalogical proofs for soundness and completeness, and a chapter outlining the connection between formal logic and computer circuits (logic gates). The biography for William of Ockham was provided by Andrew Jeffery and used with his permission. The resulting text is licensed under a Creative Commons Attribution 4.0 license. There are \href{https://github.com/OpenLogicProject/OpenLogic/wiki/Other-Logic-Textbooks}{several other} ``remixes'' of \emph{forall x} available freely online.

\paragraph{Notes for instructors} This book was designed for a quarter-long (10 weeks) introduction to formal logic. In my classes, I (Davis A. Smith) cover all modules, spending 1 week per module. 
The CC BY license gives you the right to download and distribute the book yourself. In order to ensure that all your students have the same version of the book throughout the term you're using it, you should upload the PDF you decide to use to your LMS rather than merely give your students a link to the source. You are also free to have the PDFs printed by your bookstore.

The syntax and proof systems are supported by Graham Leach-Krouse's free, online logic teaching software application \href{https://carnap.io}{Carnap.io}. I recommend that this be included in your course as a course companion. It allows for submission and automated marking of exercises such as symbolization, truth tables, and natural deduction proofs.