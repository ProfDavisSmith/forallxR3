%!TEX root = forallxd.tex
\part{Arguments and Logic}
\label{ch.intro}
\addtocontents{toc}{\protect\mbox{}\protect\hrulefill\par}


\chapter{Part 1: Arguments and Propositions}
\section{Part 1.1: Arguments: What is logic?}
\label{s:Part 1.1: Arguments: What is logic?}

The common-sense, everyday, way to tell whether a claim is true or false is to look at the reasons for or against it. Sometimes our observations give us good reasons. For example, I have a good reason for thinking my bicycle has a flat tire when I see the tire sagging on the rim or hear air hissing out of the tube. But often the business of identifying and evaluating reasons is a bit more involved. Logic is the business of identifying and evaluating reasons. You do this all the time. You give yourself reasons to choose one shirt over another for a job interview, you reason your way through making a choice about which classes to take, you believe certain things based on evidence, and you give reasons for why something isn't your fault (when you make excuses). In all of these cases and more, you are making \glspl{argument}.

This is very different from how you might use the term ‘argument'. In everyday language, we sometimes use the word ‘argument’ to talk about belligerent shouting matches. If you and a friend have an argument in this sense, things are not going well between the two of you. Logic is not concerned with such teeth-gnashing and hair-pulling. They are not arguments, in our sense; they are just disagreements. For a humorous example of how arguments and disagreements differ, take a look at this funny exchange from Monty Python:\autocite{ArgumentClinic}

\factoidbox{\begin{multicols}{2}
Man: Is this the right room for an argument?\\
Other Man:(John Cleese) I've told you once.\\
Man: No you haven't.\\
Other Man: Yes I have.\\
M: When?\\
O: Just now.\\
M: No you didn't!\\
O: Yes I did!\\
M: You didn't!
\ldots\\
O: Oh I'm sorry, is this a five minute argument, or the full half hour?\\
M: Ah! (taking out his wallet and paying) Just the five minutes.\\
O: Just the five minutes. Thank you.\\
O: Anyway, I did.\\
M: You most certainly did not!\\
O: Now let's get one thing quite clear: I most definitely told you!\\
M: Oh no you didn't!\\
O: Oh yes I did!
\ldots\\
M: Oh look, this isn't an argument!\\
(pause)\\
O: Yes it is!\\
M: No it isn't!\\
(pause)\\
M: It's just contradiction!\\
O: No it isn't!\\
M: It IS!\\
O: It is NOT!
\ldots\\
M: (exasperated) Oh, this is futile!!\\
(pause)\\
O: No it isn't!\\
M: Yes it is!\\
(pause)\\
M: I came here for a good argument!\\
O: AH, no you didn't, you came here for an argument!\\
M: An argument isn't just contradiction.\\
O: Well! it CAN be!\\
M: No it can't!\\
M: \emph{An argument is a connected series of statements intended to establish a proposition.}\\
O: No it isn't!\\
M: Yes it is! ‘tisn't just contradiction.\\
O: Look, if I \emph{argue} with you, I must take up a contrary position!\\
M: Yes but it isn't just saying ‘no it isn't'.\\
O: Yes it is!
\ldots\\
M: No it ISN'T! \emph{Argument is an intellectual process.} Contradiction is just the automatic gainsaying of anything the other person says.\\
O: It is NOT!\\
M: It is!
\ldots 
\end{multicols}}

An argument is a reason for taking something to be true. \Glspl{argument} are made out of two or more claims, one of which is a \gls{conclusion}. The conclusion is the claim the argument purports to give a reason for believing. The other claims are the \glspl{premise}. The premises of an argument taken together are offered as a reason for believing its conclusion. In the above exchange, the person looking for an argument claims that "An argument is a collected series of statements to establish a definite proposition". This is a close approximation to how philosophers use the term, but a better, more exact definition is:
\begin{center}
An argument is a collected series of propositions intended to establish others in the series.
\end{center}
We will be returning to what, exactly, propositions are in a moment but the propositions (statements, in this case) which support others are our premises. The one being supported is the conclusion.  This is very different from a disagreement; in the above discussion, the contrarian (‘O') is conflating disagreement with argument.  A disagreement is where you have two things (often people) which do not agree about the truth of some proposition.  Often, in at least a reasonable disagreement, one or both of the parties will give arguments in support of their side. They will give reasons for why they are correct and the opposing side is, thereby, wrong. So, an argument, as we will understand it, is something more like this:
	\begin{earg}\label{argButlerGardner}
		\item[] Either the butler or the gardener did it.
		\item[] The butler didn't do it.
		\item[\therefore] The gardener did it.
	\end{earg}
We here have a series of sentences. The three dots on the third line, \gls{therefore}, of the argument are read ‘therefore.’ They indicate that the final sentence expresses the conclusion of the argument. The two sentences before that are the premises of the argument. If you believe the premises and you think the conclusion follows from the premises (that the argument, as we will say, is valid), then this (perhaps) provides you with a reason to believe the conclusion.

This is the sort of thing that logicians are interested in.  Without arguments, we would be unable to tell which side is correct. In fact, without arguments, we wouldn't be able to decide anything (all of our choices would be completely random). We will say that an argument is any collection of premises, together with a conclusion. 

Logic, as a discipline, is the study of arguments and structures of reasoning. In this class, you will learn and apply various simple reasoning structures (called ‘rules of inference') and learn to identify when two statements mean the same thing and how to move between these statements.

\newglossaryentry{therefore}{
name = {\therefore},
text = \therefore,
description = {A shorthand for the various ways in ordinary language we indicate the conclusion to an argument or line of reasoning, \gls{conclusion indicator word}s}
} 

\subsection{What are we doing in this first module?}

This first module discusses some basic logical notions that apply to arguments in a natural language like English (these same principles apply to all natural languages and to many artificial languages (sometimes called ‘Conlangs' for constructed languages)). It is important to begin with a clear understanding of what arguments are and of what it means for an argument to be valid. Later (in Module \ref{ch.symbolizing}), we will translate arguments from English into a formal language. This move from a natural language to a formal, logical, one is important. The formal language allows us to look at how the argument was reasoned and avoid the biases we are prone to have when we know what was reasoned.  We want formal validity, as defined in the formal language, to have at least some of the important features of natural-language validity so that it translates over to our every-day life.

\subsection{The general structure of an argument}

In the butler and gardener examples above, we used individual sentences to express both of the argument’s premises, and we used a third sentence to express the argument’s conclusion. Many arguments are expressed in this way, especially when clarity is more important than efficiency or rhetoric, but a single sentence can contain a complete argument. Consider:
	\begin{quote}
		 The butler has an alibi; so they cannot have done it.
	\end{quote}
This argument has one premise followed by a conclusion. There are likely some \glspl{implied premise} such as "if the butler has an alibi, then they cannot have done it". In a formal language, such as the one you will be learning in this class, those will need to be explicit. Many arguments, such as that one,  start with premises, and end with a conclusion, but not all of them. The argument with which this section began might equally have been presented with the conclusion at the beginning, like so:
	\begin{quote}
		The gardener did it. After all, it was either the butler or the
		gardener. And the butler didn't do it.
	\end{quote}
So, in this case, the premises are "it was either the butler or the gardener" and "the butler did it". Here, we don't have, or don't need, any implied premises. Equally, it might have been presented with the conclusion in the middle:
	\begin{quote}
		The butler didn't do it. Accordingly, it was the gardener,
		given that it was either the gardener or the butler.
	\end{quote}
When approaching an argument, we want to know whether or not the conclusion follows from the premises. So the first thing to do is to separate out the conclusion from the premises. As a guide, these words are often used to indicate an argument's conclusion:
	\begin{center}
		so, therefore, hence, thus, accordingly, consequently
	\end{center}
For this reason, they are sometimes called \glspl{conclusion indicator word}. By contrast, these expressions are \glspl{premise indicator word}, as they often indicate that we are dealing with a premise, rather than a conclusion:
	\begin{center}
		since, because, given that
	\end{center}
But in analysing an argument, there is no substitute for a good nose.


\newglossaryentry{implied premise}
{
name=implied premise,
description={A premise of an argument which is not explicitly stated but rather is hinted at or inferred by the context}
}


\section{Part 1.2 Sentences and Propositions}
\label{Some Basics from Philosophy of Language}

To be perfectly general, we can define an \gls{argument} as a series of sentences. The sentences at the beginning of the series are premises. The final sentence in the series is the conclusion. If the premises are true and the argument is a good one, then you have a reason to accept the conclusion.

In logic, we are only interested in sentences that can figure as a premise or conclusion of an argument, i.e., sentences that can be true or false. This helps us avoid the pulls of rhetoric and certain kinds of fallacious reasoning. We also want to make sure that we all have attached the same meaning to a sentence, as to avoid the potential confusion resulting from us talking past each other. So, in general, logic is focused on propositions.

This is where we get into a very basic introduction into Philosophy of Language. This area of philosophy deals with things such as meaning, reference, and so forth (Philosophy of Language is what I mostly work in, it deals with semantics while Linguistics deals with syntax, but the fields overlap quite a bit). Even when we aren't trying very hard, we can use words and sentences in a ton of different ways and there seems to be some vagueness or ambiguity in natural languages. For example, I can easily make examples where sentences might have two or more different interpretations, like double or triple entendres. \\
\begin{earg}
\item[\ex{ambi1}] GOP grills the IRS chief over lost emails.
\item[\ex{ambi2}] If your dog poops, you must put it in the trash can.
\item[\ex{ambi3}] A woman gives birth in the UK every 48 seconds.
\item[\ex{ambi4}] People wanted for pickling and canning.
\end{earg}
All four of the above examples have at least 2 different interpretations and, in each case, one seems true while the other seems false. In Example \ref{ambi1}, it could be interpreted to mean that the GOP are having an IRS chief BBQ over a server or it could mean that they are harshly questioning the chief. In Example \ref{ambi2}, one way to understand this is that it's telling you to put your dog in the garbage can, while another is telling you to put the poop therein. In Example \ref{ambi3}, we could say that there's a single woman who is having a baby every 48 seconds (she must be tired) or that some woman, not necessarily the same woman, is having a child in that time. In Example \ref{ambi4}, it could be that we have a cannibal looking to store their meats for the winter using canning and fermenting techniques or it could be that a food fermentation business is looking to hire on some more people in those departments. Given these examples, and how often we misunderstand each other on a daily basis,  it might be tempting, therefore, to think that the truth of a sentence must be relative to its interpretation.  In an even more robust example, imagine the following:
\factoidbox{Suppose that we all collectively switched the sorts of things the words ‘dog' and ‘cat' pick-out. So, the word ‘dog' now is used to point out the meowing critters and the word ‘cat' refers to the barking ones. In this case, it would seem that the sentence ‘dogs are canines' would be false and ‘dogs are feline' would be true. If we, again, flipped the meanings of the words ‘feline' and ‘canine', we would get that ‘dogs are canines' as true, but for a totally different reason than it was originally.}
But does this make truth open for interpretation? Well, no, but in a sense, yes. When we look at a language, from one perspective, we notice that things like words and sentences are nothing more than characters on a screen/page, random collections of sounds, or certain structured gestures (in the case of sign languages). Those things, on their own, don't have meaning. There must be something extra, beyond the sound and signs, which has the meaning.  Philosophers call the meaning behind a sentence a \gls{proposition}. A proposition, itself, is not a sentence or a word, rather it's the meaning behind the sentences.\\

\newglossaryentry{proposition}
{
name=proposition,
description={The meaning behind a declarative sentence or expression}
}

\begin{center}
\begin{tabular}{c|c|c}\hline
Sentence&Language&Proposition\\\hline
Snow is white&English&that snow is white\\
Schnee ist Weiss&German&that snow is white\\
Nix alba est&Latin&that snow is white\\
La neige est blanche&French&that the snow is white\\
\hline
\end{tabular}
\end{center}
All of the above examples are different sentences, made clear because they are in different languages, but they all express the same proposition. The truth of a sentence is relative to the truth of the proposition attached to it, propositions are the things which are true or false, and then the truth of the proposition is determined by its correspondence with reality. Translators, at least the good ones, often take the sentence in one language, figure out the proposition connected to it, and then express the same proposition in the necessary language.

\subsection{The Move to Propositions}
So, the proposition expressed by a sentence is not itself a linguistic thing. Propositions themselves don't have meaning, but rather they are the meaning behind a sentence. For a bit of language to be open to interpretation is for us to be able to attach different propositions (meanings) to it. But the meanings themselves are not open to further interpretation. And it is the proposition, what is meant by the sentence, which makes the statements, sentences, true or false. So, when I speak of arguments consisting of premises (the supporting evidence and their arrangement in the argument), I am talking about the core meaning behind the sentence, not the sentence itself. If we misinterpret the sentence, then we haven’t yet gotten on to the claim being made and hence probably don’t fully understand the argument.

You should not confuse the idea of a sentence which can express propositions with the difference between fact and opinion. Often, sentences in logic will express things that would count as facts— such as ‘Kierkegaard was a hunchback’ or ‘Kierkegaard liked almonds’ but they can also express things which you might think of as matters of opinion—such as, ‘almonds are tasty.’ In other words, a sentence is not disqualified from being part of an argument because we don’t know if it is true or false or because its truth or falsity is a matter of opinion. If it is the kind of sentence that could be true or false, then it can play the role of premise or conclusion.

Also, there are things that would count as ‘sentences’ in a linguistics or grammar course which do not necessarily express propositions. Here are just a few:

\subsection{Questions} In a grammar class, `Are you sleepy yet?' would count as an interrogative sentence. Although you might be sleepy or you might be alert, the question itself is neither true nor false. For this reason, questions will not count as sentences in logic. Suppose you answer the question: `I am not sleepy.' This is either true or false, and so it is a sentence in the logical sense. Generally, \emph{questions} will not count as sentences, but \emph{answers} will. `What is this course about?' is not a sentence (in our sense). `No one knows what this course is about' is a sentence. 

\subsection{Imperatives} Commands are often phrased as imperatives like `Wake up!', `Sit up straight', and so on. In a grammar class, these would count as imperative sentences. Although it might be good for you to sit up straight or it might not, the command is neither true nor false. Note, however, that commands are not always phrased as imperatives. In the right context, the sentence `you will respect my authority' \emph{is} either true or false---either you will or you will not---and so it counts as a sentence in the logical sense. In another context, however, it is an imperative, so it is not the sort of thing which is true or false. Understanding which is the correct interpretation is essential to understanding language.  

\subsection{Exclamations} `Ouch!' is sometimes called an exclamatory sentence, but it is neither true nor false. We will treat `Ouch, I hurt my toe!' as meaning the same thing as `I hurt my toe.' The `ouch' does not add anything that could be true or false. Certain phrases use words and bits of language which in other contexts would amount to sentences but are in fact exclamations. For example, "you jerk!" could be easily seen as a colloquial shortening of "you are a jerk!" which is something which can be true or false, either the \emph{you} in question is a jerk or they are not. But it could also be seen as an exclamation without any additional content. As before, the context very much matters. 

\practiceproblems
\problempart
\label{pr.IDargs}
For each, identify whether it is an argument, in the logical sense.
\begin{enumerate}
\item Betty and Sue are worried about their son, Max, who is late getting home. Betty thinks that they should get in the car and drive over to his friend's house while Sue thinks that they should give Max a little more time, as he could of lost track of it. 
\item Bernie and Tommy disagree about whether they can afford to go on vacation to the ski slopes in Canada this winter. Bernie says ``Of course we can afford it. After all, you and I are both employed, we have a ton of paid time-off built up, and I have even laid out the budget here.''
\item Jesus loves me, this I know, because the Bible tells me so. 
\item Ashley and Abbott are unsure about who said the famous line ``iacta alea est''. Kirsten chimes in from across the room: ``It was Julius Caesar!''
\item If absence makes the heart grow fonder, then attendance makes the heart grow colder. Obviously, you have been in my presence for too long. 
\item I know. You know I know. I know you know I know. We know Henry knows, and Henry knows we know it. We're a knowledgeable family.
\end{enumerate}

\problempart
\label{pr.IDprops}
For each, identify whether it expresses a proposition. If there are multiple sentences, identify which, if any, express a proposition.
\begin{enumerate}
\item Mom asked me whether I would be home in time for dinner. I replied ``yes.''
\item What time do you want me to pick you up from the mall? 
\item Oh, Freddy, \emph{you} don't think I'm a heartless guttersnipe, do you?
\item A young boy says that he hates spinach. Later he says that he is glad he hates spinach. Why? Because ``if I liked spinach, I would eat it, and it's just gross!''
\item I believe that Julius Caesar crossed the Rubicon.
\item Max is such a jerk. I mean, how could anyone like him? 
\end{enumerate}


\chapter{Part 2: Types of Arguments}
\label{s:Valid}

\section{Part 2.1: Consequence and validity}

In Part 1, we talked about arguments, i.e., a collection of sentences (the premises), followed by a single sentence (the conclusion). We said that some words, such as “therefore,” indicate which sentence is supposed to be the conclusion. “Therefore,” of course, suggests that there is a connection between the premises and the conclusion, namely that the conclusion follows from, or is a consequence of, the premises. This notion of consequence is one of the primary things logic is concerned with. Referencing back, one might even say that logic is the science of what follows from what. Logic develops theories and tools that tell us when a proposition follows from some others. Take, for example, the argument which I used in Part 1:
\begin{earg}
	\item[] Either the butler or the gardener did it.
	\item[] The butler didn't do it.
	\item[\therefore] The gardener did it.
\end{earg}
We don’t have any context for what the sentences in this argument refer to. Perhaps you suspect that “did it” here means “was the perpetrator” of some unspecified crime (we don't necessarily know what the correct proposition is for that sentence in this context because we don't know the context). You might imagine that the argument occurs in a mystery novel or TV show, perhaps spoken by a detective working through the evidence. But even without having any of this information, you probably agree that the argument is a good one because of the basic structure which is used. In this sense, whatever the premises refer to, if they are both true, the conclusion must be true as well. If the first premise is true, i.e., it’s true that “the butler did it or the gardener did it,” then at least one of them “did it,” whatever that means. And if the second premise is true, then the butler did not “do it.” That leaves us with only one option: “the gardener did it” must be true. Here, the conclusion follows from the premises. We call arguments that have this property \gls{valid}.
By way of contrast, consider the following argument:
\begin{earg}\label{argMaidDriver}
	\item[] If the driver did it, the maid didn't do it.
	\item[] The maid didn't do it.
	\item[\therefore] The driver did it.
\end{earg}
We still have no idea what is being talked about here. But, again, you probably agree that this argument is different from the previous one in an important respect. If the premises are true, it is not guaranteed that the conclusion is also true. (if you disagree, then you have fallen victim to a very common logical fallacy, one which will become clear as we progress through this textbook). The premises of this argument do not rule out, by themselves, that someone other than the maid or the driver “did it.” So there is a case where both premises are true, and yet the driver didn’t do it, i.e., the conclusion is not true. For example, suppose that the context is that they are wondering who stole the last cookie from the jar. In this case, if the driver ‘did it', as in stole the cookie, then the maid couldn't have stolen it, but there was a third person, the gardener, who was feeling peckish, has stolen the cookie. As a result, the maid didn't ‘do it', the driver didn't, but the gardener did. In this second argument, the conclusion does not follow from the premises. If, like in this argument, the conclusion does not follow from the premises, we say it is invalid. As we will see later, a very easy test for validity in a natural language, like English, is to see whether you can concoct a story where the premises are true and the conclusion is false. If you can do this, then the argument is invalid and, to make it valid, you would need to include premises which remove the possibility of such a case.

\section{Part 2.2 Cases and Types of Validity}
Towards the end of the previous sub-part, I gave a basic reason why the example argument wasn't valid; namely, I was able to give a case in which the premises are true and in which the conclusion is not. This was the scenario where neither the driver nor the maid did it, but some third person, the gardener, did. Let’s call such a case a counterexample to the argument. If there is a counterexample to an argument, the conclusion cannot be a consequence of the premises. For the conclusion to be a consequence of the premises, the truth of the premises must guarantee the truth of the conclusion. If there is a counterexample, the truth of the premises does not guarantee the truth of the conclusion. As logicians, we want to be able to determine when the conclusion of an argument follows from the premises. And the conclusion is a consequence of the premises if there is no counterexample—no case where the premises are all true but the conclusion is not. This motivates a definition:
\factoidbox{A sentence A is a consequence of sentences $B_1$, \ldots,$ B_n$ if and only if there is no case where $B_1$, \ldots,$ B_n$ are all true and A is not true.\\
We then also say that A follows from $B_1$, \ldots,$ B_n$ or that $B_1$, \ldots,$ B_n$ entail A.}
This “definition” is incomplete: it does not tell us what a “case” is or what it means to be “true in a case.” So far we’ve only seen an example: a hypothetical scenario involving three people. Of the three people in the scenario—a driver, a maid, and some gardener—the driver and maid didn’t do it, but the gardener did. In this scenario, as described, the driver didn’t do it, and so it is a case in which the sentence “the driver did it” is not true. The premises of our second argument are true, but the conclusion is not true: the scenario is a counterexample. I also said that arguments where the conclusion is a consequence of the premises are called valid, and those where the conclusion isn’t a consequence of the premises are \gls{invalid}. Since we now have at least a first stab at a definition of “consequence,” we’ll record this:
\factoidbox{An argument is valid if and only if the conclusion is a consequence of the premises, i.e. it is not possible to make a counterexample.}
This definition also gives us what it means for an argument to be invalid:
\factoidbox{An argument is invalid if and only if it is not valid, i.e., it has a counterexample.}
Logicians are in the business of making the notion of “case” more precise, and investigating which arguments are valid when “case” is made precise in one way or another. If we take “case” to mean “hypothetical scenario” like the counterexample to the second argument, it’s clear that the first argument counts as valid. If we imagine a scenario in which either the butler or the gardener did it, and also the butler didn’t do it, we are automatically imagining a scenario in which the gardener did it. So any hypothetical scenario in which the premises of our first argument are true automatically makes the conclusion of our first argument true. This makes the first argument valid.
Making “case” more specific by interpreting it as “hypothetical scenario” is an advance. But it is not the end of the story. The first problem is that we don’t know what to count as a hypothetical scenario. Are they limited by the laws of physics? By what is conceivable, in a very general sense? What answers we give to these questions determine which arguments we count as valid.
\subsection{Nomological Validity}
Suppose that we said that we are limited to the laws of physics. Consider the following argument:
\begin{earg}
\item[]The spaceship Rocinante took six hours to reach Jupiter from Tycho space station.
\item[\therefore] The distance between Tycho space station and Jupiter is less than 14 billion kilometers.
\end{earg}
A counterexample to this argument would be a scenario in which the Rocinante makes a trip of over 14 billion kilometers in 6 hours, exceeding the speed of light. Since such a scenario is incompatible with the laws of physics, there is no such scenario if hypothetical scenarios have to conform to the laws of physics. This would mean that the above argument is nomologically valid. We can't come up with a counterexample which conforms to our laws of physics. If we said, however, that counterexamples (hypothetical scenarios) are not limited by the laws of physics, there is a counterexample: a scenario where the Rocinante travels faster than the speed of light. This means that the argument is not valid (full stop) but rather is valid with some assumptions. There could be an implied premise, which needs to be made explicit, that the laws of physics are accurate and that they exclude the possibility of traveling faster than the speed of light.
\subsection{Conceptual Validity}
Suppose that we said that hypothetical scenarios are limited to what we can imagine (conceive) and consider another argument:
\begin{earg}
\item[]Priya is an ophthalmologist.
\item[\therefore] Priya is an eye doctor.
\end{earg}
If we’re allowing only conceivable scenarios, this is also a valid argument. If you imagine Priya being an ophthalmologist, you thereby imagine Priya being an eye doctor. That’s just what “ophthalmologist” and “eye doctor” mean. A scenario where Priya is an ophthalmologist but not an eye doctor is ruled out by the conceptual connection between these words. We call this sort of validity conceptual validity because the truth of the conclusion is entailed by the meaning of, the proposition expressed by, the premise. Conceptually valid arguments tend to be valid (full stop) but there are some arguments (in fact most arguments) which are valid (full stop) but not necessarily conceptually valid. Such arguments are valid without relying on the definition of some word or phrase and end to be far stronger and better in reasoning.
\section{Part 2.3 Formal Validity}
One distinguishing feature of logical consequence, however, is that it should not depend on the content of (the concepts in) the premises and conclusion, but only on their logical structure, the general shape or style the argument comes in. In other words, as logicians, we want to develop a theory that can make even more fine-grained distinctions. For instance, both of these arguments are valid:\\
\begin{tabular}{p{4.8cm}p{4.8cm}}
Argument \exarg{priya1}&Argument \exarg{priya2}\\
Either Priya is an 
ophthalmologist or a dentist.&Either Priya is an 
ophthalmologist or a dentist.\\
Priya isn’t a dentist.&Priya isn’t a dentist.\\
\therefore Priya is an eye doctor.&\therefore Priya is an ophthalmologist.
\end{tabular}\\
The first argument, Argument \ref{priya1}, is valid because of the implied premise about the meaning of the word “ophthalmologist”. It is not possible to come up with a case where the premises are true and the conclusion is false because of the concepts involved. Despite looking very similar to the first, Argument \ref{priya2} does not rely on the concepts in the premises. The second argument is formally valid. This is not because it's valid because of its attire and proper decorum, rather it's valid because of the form (shape or structure) the argument has. We can describe the “form” of this argument as a pattern, something like this:
\begin{earg}
\item[] Either A is an X or a Y.
\item[] A isn’t a Y.
\item[\therefore] A is an X .
\end{earg}
Here, A, X , and Y are placeholders for appropriate expressions that, when substituted for A, X , and Y , turn the pattern into an argument consisting of sentences. Replacing the key terms or phrases with these placeholders is a really good strategy to be able to strip back the ‘what was said' and look exclusively at the ‘how it was reasoned'. There are many, many, different arguments which all use this same general pattern. Once we make a conclusion about the general pattern, that conclusion trickles down to all of the instances of that pattern. For example,
\begin{earg}
\item[]Either Mei is a mathematician or a botanist.
\item[]Mei isn’t a botanist.
\item[\therefore] Mei is a mathematician.
\end{earg}
is an argument of the same form, and we know that it is formally valid for the same reason as the general case. Argument 1, on the other hand, does not use the same form: we would have to replace Y by different expressions (once by “ophthalmologist” and once by “eye doctor”) to obtain it from the pattern. Its form is this:
\begin{earg}
\item[]Either A is an X or a Y.
\item[]A isn’t a Y.
\item[\therefore] A is a Z.
\end{earg}
In this pattern we can replace X by “ophthalmologist” and Z by “eye doctor” to obtain the original argument, which was valid because of the concepts involved. But here is another argument of the same form:
\begin{earg}
\item[]Either Mei is a mathematician or a botanist.
\item[]Mei isn’t a botanist.
\item[\therefore] Mei is an acrobat.
\end{earg}
This argument is clearly not valid, since we can imagine a mathematician named Mei who is not an acrobat. This means that the above argument structure is not formally valid and this conclusion trickles down to the various instances of the form. Our strategy as logicians will be to come up with a notion of “case” on which an argument turns out to be valid if it is formally valid. We can say that the arguments are valid, in this sense, when there are no cases where the premises of that form are all true and the conclusion (of that form) is false. Similarly, we can say that an argument is invalid, in this sense, when there is a case where the premises of this form are all true and the conclusion is false. We could say that formal validity trickles down to the instances of the form where as formal invalidity floats up to the general form.

When we consider cases of various kinds in order to evaluate the validity of an argument, we will make a few assumptions/restrictions. First, we will only consider arguments of the same form as the one in question. It wouldn't make sense to judge an argument of one form based on the features of an argument of a radically different form. The second assumption is that when we are considering a hypothetical scenario, ever premise of the argument is true. So, for example, in the mathematician vs botanist example, the hypothetical scenario will be one where Mei is either a mathematician or a botanist (at least one of them) and one where she is not a botanist. This is so that we can test to see whether the form actually entails the conclusion. And the third assumption is that the conclusion to the argument is false. If we can come up with an argument of a particular form which does have a counter example, then that argument is invalid. If we can't, then the argument is valid (we will be exploring more on this in Module~\ref{ch.TruthTables}).
\section{Part 2.4 Soundness}
Before we go on and execute this strategy, a few clarifications. Arguments in our sense, as conclusions which (supposedly) follow from premises, are of course used all the time in everyday and scientific discourse. When they are, arguments are given to support or even prove their conclusions. Now, if an argument is valid, it will support its conclusion, but only if its premises are all true. Validity rules out the possibility that the premises are true and the conclusion is not true at the same time. It does not, by itself, rule out the possibility that the conclusion is not true, period. In other words, it is perfectly possibly for a valid argument to have a conclusion that isn’t true!\\
Consider this example,
\begin{earg}
\item[]Oranges are either fruit or musical instruments.
\item[]Oranges are not fruit.
\item[\therefore] Oranges are musical instruments.
\end{earg}
The conclusion of this argument is ridiculous. Nevertheless, it follows from the premises. If both premises are true, then the conclusion just has to be true. So the argument is valid.
Conversely, having true premises and a true conclusion is not enough to make an argument valid. Consider this example:
\begin{earg}
\item[]London is in England.
\item[]Beijing is in China.
\item[\therefore] Paris is in France.
\end{earg}
The premises and conclusion of this argument are, as a matter of fact, all true, but the argument is invalid. If Paris were to declare independence from the rest of France, then the conclusion would no longer be true, even though both of the premises would remain true. Thus, there is a case where the premises of this argument are true without the conclusion being true. So the argument is invalid.

The important thing to remember is that validity is not about the actual truth or falsity of the sentences in the argument. It is about whether it is possible for all the premises to be true and the conclusion to be not true at the same time (in some hypothetical case). What is in fact the case has no special role to play; and what the facts are does not determine whether an argument is valid or not. Nothing about the way things are can by itself determine if an argument is valid. It is often said that logic doesn’t care about feelings. Actually, it doesn’t care about facts, either.
When we use an argument to prove that its conclusion is true, then, we need two things. First, we need the argument to be valid, i.e., we need the conclusion to follow from the premises. But we also need the premises to be true. We will say that an argument is sound if and only if it is both valid and all of its premises are true. The flip side of this is that when you want to rebut an argument, you have two options: you can show that (one or more of) the premises are not true, or you can show that the argument is not valid. Logic, however, will only help you with the latter!
\section{Part 2.5 Deductive vs Inductive Arguments}
Strictly speaking, only what we call \glspl{deductive argument} can be valid. As I have mentioned previously, an argument is all about intent. The premises are intended to support the conclusion. Deductive arguments are those where the intent is to guarantee the conclusion. When there is that intent, we can and should judge it according to validity. But there are many arguments out there where the intent is not to guarantee the conclusion, but rather to merely make it more likely. These arguments, by their very nature, are going to be invalid but they are still good and have practical use. These arguments are called \glspl{inductive argument}. This is because, generally, they go from a bunch of particular cases and generalize to something larger. Deductive arguments, on the other hand, tend to go from general principles and then reduce (deduce) to particular instances. To see the difference, take a look at these examples:

\begin{tabular}{p{4.8cm}|p{4.8cm}}\hline
Argument \exarg{linecooktime1}:&Argument \exarg{linecooktime2}:\\
Sam is a line cook.&Sam is a line cook.\\
All line cooks have good time management skills.&Most line cooks have good time management skills.\\
Therefore, Sam has good time management skills&Therefore, Sam probably has good time management skills.\\
\end{tabular}\\

Both of these arguments, \ref{linecooktime1} and \ref{linecooktime2}, start with the same premise, "Sam is a line cook" and they come to \emph{similar} conclusions, but the certainty expressed by these arguments is very different. In argument \ref{linecooktime1}, the conclusion means something certain or absolute while argument \ref{linecooktime2} has a conclusion with room for a little doubt; maybe Sam is a line cook who hasn't been put under the pressure to need to develop the time management skills yet. The difference in the tone of the conclusions gives us a hint about the intent behind these arguments. Argument \ref{linecooktime1} has the intent to guarantee the conclusion while argument \ref{linecooktime2} merely wants to make it more likely. This, roughly, makes argument \ref{linecooktime1} deductive and is thereby judged by validity and the second, argument \ref{linecooktime2} is inductive and is judged by strength. It doesn't make sense to claim that a deductive argument is strong because that standard just doesn't apply. 

Unlike validity, strength does come in degrees. With validity, it's all or nothing, so to speak. With strength, one argument could be stronger than another and both still be strong. The gauge here is how likely the conclusion is relative to the premises. How inductive arguments can have different strengths (meaning that one is stronger than another) is best introduced with examples:

\begin{tabular}{p{4.8cm}|p{4.8cm}}\hline
Argument \exarg{linecook1}:&Argument \exarg{linecook2}:\\
Sam is a line cook.&Sam is a line cook.\\
Line cooks generally have good kitchen skills.&Line cooks generally aren't paid well.\\
Therefore, Sam can probably cook well&Therefore, Sam is probably a billionaire.\\
\end{tabular}\\

\newglossaryentry{deductive argument}
{
name=deductive argument,
description={An argument such that the intent behind it is to guarantee the conclusion. There is no ‘wiggle room'. In general, deductive arguments move from broad general principles and to more particular instances.},
plural=deductive arguments
}

\newglossaryentry{inductive argument}
{
name=inductive argument,
description={An argument such that the intent behind it is \emph{merely} to make the conclusion more likely, assuming the truth of the premises. There is some ‘wiggle room'. In general, inductive arguments move from a collection of particular instances and then use those to support the truth of some general principle},
plural=inductive
}


In argument \ref{linecook1}, we can see that the arguer, through the argument, doesn't intend to make a claim that couldn't be false. Rather, they are trying to make the claim that it's likely that Sam can cook well. While this argument is invalid (one can easily come up with a case where the premises are true and the conclusion is false, such as a scenario where Sam was just hired by a seafood buffet in Arizona), we are willing to say that the premises make the conclusion more likely than otherwise. When the premises do this, we say that the inductive argument is \gls{strong inductive argument}. 

\newglossaryentry{strong inductive argument}
{
name=strong,
description={An inductive argument such that the premises, assuming that they are true, make the conclusion more likely than otherwise. This comes in degrees; that is, some strong inductive arguments may be stronger than others}
}

In argument \ref{linecook2}, the conclusion is very unlikely given the premises, so we would say that the argument is \gls{weak inductive argument} (not strong). These arguments, however, have different conclusions. It is certainly possible for two inductive arguments to come to the same conclusion and have different levels of strength. Take a look at these arguments: 

\newglossaryentry{weak inductive argument}
{
name=weak,
description={An inductive argument such that the premises, assuming that they are true, \emph{fail} make the conclusion more likely than otherwise. This comes in degrees; that is, some weak inductive arguments may be weaker than others}
}


\begin{tabular}{p{4.8cm}|p{4.8cm}}\hline
Argument \exarg{raven1}:&Argument \exarg{raven2}:\\
All the ravens my son has seen have had black feathers.&All the ravens Patty has seen have had black feathers.\\
My son is 5 years old.&Patty is an ornithologist.\\
Therefore, all ravens likely have black feathers.&Therefore, all ravens likely have black feathers.\\
\end{tabular}\\

As you can see, both of these arguments come to the same conclusion, namely "all ravens likely have black feathers" but the strength of these arguements differ (because of who is cited to support the claim). Argument \ref{raven1} is weak because, given the premises, the conclusion is less likely than otherwise. How much could the average 5 year old know about ravens?  Argument \ref{raven2}, on the other hand, is significantly stronger. If an expert ornithologist (a person who studies birds) says that all ravens have black feathers, while there is still the possibility that they are wrong, we could still take that information to the bank (there is a family of white feathered ravens which frequents some of the beaches in western Washington, USA).  

On a practical level, you should temper your belief that the conclusion of an inductive argument is true according to the strength of the argument. If some actor, with no education in anthropology, archeology, or history appears on your TV claiming that the great pyramids in Egypt were actually landing pads for extraterrestrial space craft, you should be very skeptical of that claim unless they add in more evidence and claims which increase the likelihood. If, on the other hand, an expert in one of those fields claims that they have found evidence that the pyramids were landing pads, we should be less skeptical. With inductive arguments, we assume the premises are true and see how likely it is that the conclusion would be true. Strong arguments are those which make it unlikely that the conclusion would be false and weak arguments are those where the conclusion would still be unlikely.  

The point of all this is that inductive arguments—even good inductive arguments—are not (deductively) valid. They are not watertight. Unlikely though it might be, it is possible for their conclusion to be false, even when all of their premises are true. In this class, we will set aside (entirely) the question of what makes for a good inductive argument. Our interest is simply in sorting the (deductively) valid arguments from the invalid ones. So: we are interested in whether or not a conclusion follows from some premises. Don’t, though, say that the premises infer the conclusion. Entailment is a relation between premises and conclusions; inference is something we do. So if you want to mention inference when the conclusion follows from the premises, you could say that one may infer the conclusion from the premises.

\turing

\practiceproblems
\problempart
\label{pr.typesofargs}
For each of the following: Identify whether the argument is inductive or deductive.
\begin{enumerate}
\item If Max goes to the party, Sally will be happy. If Sally is happy, Klaus will want to know why. Therefore, if Max goes to the party, Klaus will want to know why Sally is happy.
\item George H.W. Bush hates broccoli. If he hates broccoli, then he likely banned it from the White House. Therefore, Bush Sr. likely banned it from the White House.
\item If the glove does not fit his hand, then he must be innocent. The glove didn't fit his hand. So, he is likely innocent. 
\item There is a 75\% chance that we landed on the moon. Therefore, we landed on the moon.
\item Does God exist? Yes! The laws of nature and the numerical constants which represent them need to be so precise that to suggest that there wasn't a creator of the universe is absurd. The only being which could have created the universe with this percision is God. 
\item If the US invades Canada, then they will want to invade Mexico. If they want to invade Mexico, then they will and will continue south. If they continue south, then they will face little resistance. The US will invade Canada. Therefore, all of the Americas will be a part of the US by the end of the century. 
\end{enumerate}

\problempart
\label{pr.validity}
All of the arguments below are deductive. For each, identify whether they are formally valid.
\begin{enumerate}
\item If Dave is out late, Sandie will worry. If Sandie is worried, then Tom will get mad. Therefore, if Dave is out late, Tom will be mad.
\item Either Patty or Max is a friend of Joe. We know that Max leaves the room whenever Joe enters it. Therefore, Patty is Joe's friend.
\item If it is snowing, then Cory will be out with his snowboard. Cory is out with his snowboard. Therefore, it's snowing. 
\item Either Sally or Gwen has a crush on Ben. We know that Gwen can't have those feelings, as she is sister. Therefore, Sally has a crush on Ben.
\item If it is raining, then Tom is not out fishing. It's raining. Therefore, Tom's not out fishing. 
\item If the US invades Canada, then the UK will be mad. If Russia is mad, then the US meddled in the Middle East. The US will either meddle in the Middle East or invade Canada. Therefore, either the UK or Russia will be mad.  
\end{enumerate}

\problempart
\label{pr.strength}
All of the arguments below are inductive. For each, gauge how strong they are.
\begin{enumerate}
\item The recipe for the brownies says to bake at 350 degrees for 30min. Max baked them at 425 degrees for 3 hours. They are likely very burnt.
\item The sign above the well-maintained theater says that they will be showing the latest Marvels movie at 10am. Therefore, they are likely showing that movie at 10am.
\item The sign above the long out-of-business theater says that they will be showing the the Wizard of OZ at 10pm. Therefore, they are likely showing that movie at 10pm. 
\item The tree in front of my house is very old and sickly. The neighbors have always hated it being there. Last night, there was a massive windstorm and the tree came crashing down. So, they used the cover of the storm to come over and chop my tree down!
\item As technology progresses, it becomes harder and harder to tell what, in films, was CGI or was actually filmed. Slowly but surely, computers will likely be able to generate scenes which we mere humans can't distinguish from reality. Therefore, live actors will likely soon need to find new jobs as voice actors.  
\item Dave and Tami can smell tobacco smoke on their 16 year old son's jacket. When Dave did his son's laundry, he found a Zippo lighter in the bottom of the washer. Therefore, he is likely smoking and they should have a chat with him.  
\end{enumerate}

\chapter{Part 3: Other Logical Notions}
In Part 2, we introduced the ideas of consequence and of valid argument. This is one of the most important ideas in logic. In this section, we will introduce some similarly important ideas. They all rely, as did validity, on the idea that sentences are true (or not) in cases. For the rest of this section, we’ll take cases in the sense of conceivable scenario, i.e., in the sense in which we used them to define conceptual validity. The points we made about different kinds of validity can be made about our new notions along similar lines: if we use a different idea of what counts as a “case” we will get different notions. And as logicians we will, eventually, consider a more permissive definition of case than we do here.
\section{3.1 Joint Possibility}
Consider these two sentences:
\begin{ebullet}
\item[B1.] Jane’s only brother is shorter than her.
\item[B2.] Jane’s only brother is taller than her.
\end{ebullet}
Logic alone cannot tell us which, if either, of these sentences is true. For that, we would need to pull out a yard stick and measure. Yet we can say that if the first sentence (B1) is true, then the second sentence (B2) must be false. Similarly, if B2 is true, then B1 must be false. There is no possible scenario where both sentences are true together. These sentences are incompatible with each other, they cannot all be true at the same time. This motivates the following definition:
\factoidbox{Sentences are jointly impossible if and only if there are no cases where they are all true together.}
B1 and B2 are jointly impossible, while, say, the following two sentences are jointly possible:
\begin{ebullet}
\item[C1.] Jane’s only brother is shorter than her.
\item[C2.] Jane’s only brother is younger than her.
\end{ebullet}
This is because there is a scenario where they are true together, and such a scenario is easily conceived. This example generalizes to the following definition
Sentences are jointly possible if and only if there are cases where they are all true together.
Note that I did not specify anything about the number of sentences. In fact, we can ask about the joint possibility of any number of sentences. For example, consider the following four sentences:
\begin{ebullet}
\item[G1.] There are at least four giraffes at the wild animal park.
\item[G2.] There are exactly seven gorillas at the wild animal park.
\item[G3.] There are not more than two martians at the wild animal park.
\item[G4.] Every giraffe at the wild animal park is a martian.
\end{ebullet}
G1 and G4 together entail that there are at least four martian giraffes at the park. This conflicts with G3, which implies that there are no more than two martian giraffes there. So the sentences G1–G4 are jointly impossible. They cannot all be true together. (Note that the sentences G1, G3 and G4 are jointly impossible. But if sentences are already jointly impossible, adding an extra sentence to the mix cannot make them jointly possible!)

\section{3.2 Necessary truths, necessary falsehoods, and contingency}
In assessing arguments for validity, we care about what would be true if the premises were true, but some sentences just must be true. Consider these sentences:
\begin{earg}
\item[\ex{rain1}] It is raining.
\item[\ex{rain2}] Either it is raining here, or it is not.
\item[\ex{rain3}] It is both raining here and not raining here.
\end{earg}
In order to know if sentence \ref{rain1} is true, you would need to look outside or check the weather channel. It is possible that it's true and it is also possible to be false. There is nothing in the structure of the sentence itself which tells us one way or the other. A sentence which is capable of being true and capable of being false (in different circumstances, of course) is called contingent. Sentence \ref{rain2}, on the other hand, is different. I don't need to peak my head out the window or check.  Regardless of what the weather is like, it is either raining or it is not. That is a necessary truth. In general, we can tell these sort of necessary truths by the structure, the form of the statement. We will be returning to these sort of claims in Module~\ref{ch.TruthTables}. There are other ones which rely on the meaning of the words involved, such as "water is H2O", this is a necessary truth but it has the same structure as "my car is silver", which is not necessary (I could have it painted). Equally, you do not need to check the weather to determine whether or not sentence \ref{rain3} is true. It must be false, simply as a matter of logic. It might be raining here and not raining across town; it might be raining now but stop raining even as you finish this sentence; but it is impossible for it to be both raining and not raining in the same place and at the same time. So, whatever the world is like, it is not both raining here and not raining here. It is a necessary falsehood.

Something might always be true and still be contingent. For instance, if there never were a time when the universe contained fewer than seven things, then the sentence ‘At least seven things exist’ would always be true. Yet the sentence is contingent: the world could have been much, much smaller than it is, and then the sentence would have been false.
\section{3.3 Necessary equivalence}
We can also ask about the logical relations between two sentences. For example:
\begin{earg}
\item[]John went to the store after he washed the dishes.
\item[]John washed the dishes before he went to the store.
\end{earg}
These two sentences are both contingent, since John might not have gone to the store or washed dishes at all. Yet they must have the same truth-value. If either of the sentences is true, then they both are; if either of the sentences is false, then they both are. When two sentences have the same truth value in every case, we say that they are necessarily equivalent.
\subsection{Summary of logical notions}
\begin{ebullet}
\item[\textbullet] An argument is valid if there is no case where the premises are all true and the conclusion is not; it is invalid otherwise.
\item[\textbullet] A necessary truth is a sentence that is true in every case.
\item[\textbullet] A necessary falsehood is a sentence that is false in every case.
\item[\textbullet] A contingent sentence is neither a necessary truth nor a necessary falsehood; a sentence that is true in some case and false in some other case.
\item[\textbullet] Two sentences are necessarily equivalent if, in every case, they are both true or both false.
\item[\textbullet] A collection of sentences is jointly possible if there is a case where they are all true together; it is jointly impossible otherwise.
\end{ebullet}

\practiceproblems
\problempart
\label{pr.EnglishTautology2}
For each of the following: Is it a necessary truth, a necessary falsehood, or contingent?
\begin{enumerate}
\item Caesar crossed the Rubicon.
\item Someone once crossed the Rubicon.
\item No one has ever crossed the Rubicon.
\item If Caesar crossed the Rubicon, then someone has.
\item Even though Caesar crossed the Rubicon, no one has ever crossed the Rubicon.
\item If anyone has ever crossed the Rubicon, it was Caesar.
\end{enumerate}

\problempart
For each of the following: Is it a necessary truth, a necessary falsehood, or contingent?
\begin{enumerate}
\item Elephants dissolve in water.
\item Wood is a light, durable substance useful for building things.
\item If wood were a good building material, it would be useful for building things.
\item I live in a three story building that is two stories tall.
\item If gerbils were mammals they would nurse their young.
\end{enumerate}

\problempart Which of the following pairs of sentences are necessarily  equivalent? 

\begin{enumerate}
\item Elephants dissolve in water.	\\
	If you put an elephant in water, it will disintegrate.
\item All mammals dissolve in water.\\		
	If you put an elephant in water, it will disintegrate.
\item George Bush was the 43rd president. \\
	 Barack Obama is the 44th president.
\item Barack Obama is the 44th president. \\
	  Barack Obama was president immediately after the 43rd president.
\item Elephants dissolve in water. 	\\	
	All mammals dissolve in water.
\end{enumerate}
\problempart Which of the following pairs of sentences are necessarily equivalent? 

\begin{enumerate}
\item  Thelonious Monk played piano.	\\
	John Coltrane played tenor sax.
\item  Thelonious Monk played gigs with John Coltrane.	\\
	John Coltrane played gigs with Thelonious Monk.
\item  All professional piano players have big hands.	\\
	Piano player Bud Powell had big hands.
\item  Bud Powell suffered from severe mental illness.	 \\
	All piano players suffer from severe mental illness.
\item John Coltrane was deeply religious.	 \\
John Coltrane viewed music as an expression of spirituality.
\end{enumerate}

\noindent \problempart Consider the following sentences: 
\begin{enumerate}%[label=(\alph*)]
\item[G1] \label{itm:at_least_four}There are at least four giraffes at the wild animal park.
\item[G2] \label{itm:exactly_seven} There are exactly seven gorillas at the wild animal park.
\item[G3] \label{itm:not_more_than_two} There are not more than two Martians at the wild animal park.
\item[G4] \label{itm:martians} Every giraffe at the wild animal park is a Martian.
\end{enumerate}

Now consider each of the following collections of sentences. Which are jointly possible? Which are jointly impossible?
\begin{enumerate}
\item Sentences G2, G3, and G4
\item Sentences G1, G3, and G4
\item Sentences G1, G2, and G4
\item Sentences G1, G2, and G3
\end{enumerate}

\problempart Consider the following sentences.
\begin{enumerate}%[label=(\alph*)]
\item[M1] \label{itm:allmortal} All people are mortal.
\item[M2] \label{itm:socperson} Socrates is a person.
\item[M3] \label{itm:socnotdie} Socrates will never die.
\item[M4] \label{itm:socmortal} Socrates is mortal.
\end{enumerate}
Which combinations of sentences are jointly possible? Mark each ``possible'' or ``impossible.''
\begin{enumerate}
\item Sentences M1, M2, and M3
\item Sentences M2, M3, and M4
\item Sentences M2 and M3
\item Sentences M1 and M4
\item Sentences M1, M2, M3, and M4
\end{enumerate}

\problempart
\label{pr.EnglishCombinations2}
Which of the following is possible? If it is possible, give an example. If it is not possible, explain why.
\begin{enumerate}
\item A valid argument that has one false premise and one true premise

\item A valid argument that has a false conclusion

\item A valid argument, the conclusion of which is a necessary falsehood

\item An invalid argument, the conclusion of which is a necessary truth

\item A necessary truth that is contingent

\item Two necessarily equivalent sentences, both of which are necessary truths

\item Two necessarily equivalent sentences, one of which is a necessary truth and one of which is contingent

\item Two necessarily equivalent sentences that together are jointly impossible

\item A jointly possible collection of sentences that contains a necessary falsehood

\item A jointly impossible set of sentences that contains a necessary truth
\end{enumerate}

\problempart
Which of the following is possible? If it is possible, give an example. If it is not possible, explain why.

\begin{enumerate}
\item A valid argument, whose premises are all necessary truths, and whose conclusion is contingent
\item A valid argument with true premises and a false conclusion
\item A jointly possible collection of sentences that contains two sentences that are not necessarily equivalent
\item A jointly possible collection of sentences, all of which are contingent
\item A false necessary truth
\item A valid argument with false premises
\item A necessarily equivalent pair of sentences that are not jointly possible
\item A necessary truth that is also a necessary falsehood
\item A jointly possible collection of sentences that are all necessary falsehoods
\end{enumerate}

\newglossaryentry{premise indicator word}
{
name=premise indicator,
description={A word or phrase such as ``because'' used to indicate that what follows is the premise of an argument},
plural=premise indicators
}

\newglossaryentry{conclusion indicator word}
{
name=conclusion indicator,
description={A word or phrase such as ``therefore'' used to indicate that what follows is the conclusion of an argument},
plural=conclusion indicators
}

\newglossaryentry{argument}
{
name=argument,
description={A connected series of sentences, divided into \gls{premise}s and \gls{conclusion}},
plural=arguments
}

\newglossaryentry{premise}
{
name=premise,
description={A sentence in an \gls{argument} other than the \gls{conclusion}, often indicated with a premise indicator},
plural=premises
}

\newglossaryentry{conclusion}
{
name=conclusion,
description={The sentence which an argument is intended to support or prove, often indicated with a conclusion indicator.}
}

\newglossaryentry{valid}
{
name=valid,
description={A property of arguments where the truth of the premises guarantee the truth of the conclusion; i.e. it is impossible for the premises to be true and the conclusion false}
}

\newglossaryentry{invalid}
{
name=invalid,
description={A property of arguments that holds when the conclusion is not a consequence of the premises; i.e. it is possible for the premises to be true and the conclusion false. This is the opposite of \gls{valid}}
}

\newglossaryentry{sound}
{
name=sound,
description={A property of arguments that holds if the argument is valid and has all true premises}
}


\newglossaryentry{joint possibility}
{
name=joint possibility,
text={jointly possible},
description={A property possessed by some sentences when they are all true in a single case}
}

\newglossaryentry{contingent sentence}
{
name=contingent sentence,
description={A sentence that is neither a \gls{necessary truth} nor a \gls{necessary falsehood}; a sentence that in some case is true and in some other case, false}
}

\newglossaryentry{necessary truth}
{
name={necessary truth},
description={A sentence that is true in every case}
}

\newglossaryentry{necessary falsehood}
{
name={necessary falsehood},
description={A sentence that is false in every case}
}


\newglossaryentry{necessary equivalence}
{
name={necessary equivalence},
text={necessarily equivalent},
description={A property held by a pair of sentences that, in every case, are either both true or both false}
}